\documentclass{article}

\usepackage[margin=2cm]{geometry}
\usepackage[T1]{fontenc}
\usepackage[parfill]{parskip}

\usepackage{multicol}
\usepackage{amsfonts}
\usepackage{amsmath}

\usepackage{graphicx}

\newcommand{\pr}{::=}
\newcommand{\nt}[1]{\langle #1 \rangle}
\newcommand{\gr}{\;|\;}
\newcommand{\cd}[1]{\text{\lstinline[columns=fixed]{#1}}}
\newcommand{\raa}{\Rightarrow}
\newcommand{\amb}{\varepsilon}

\usepackage[only,llbracket,rrbracket]{stmaryrd}
\newcommand{\lbr}{\llbracket}
\newcommand{\rbr}{\rrbracket}
\newcommand{\typ}[1]{\lbr\, \boxed{#1} \,\rbr}
\newcommand{\typx}[1]{\lbr\, #1 \,\rbr}

\usepackage{listings}
\usepackage[dvipsnames]{xcolor}

\lstdefinestyle{minilisp}{
  commentstyle=\color{codegreen},
  keywordstyle=\color{OrangeRed},
  basicstyle=\ttfamily,
  identifierstyle=\color{Cyan}
}

\lstdefinelanguage{Minilisp}{
  language=Lisp,
  morekeywords={lambda,if,else,let*,t,f,add1,sub1,sqrt,expt,
               not,or,and,letrec,head,tail,length,reverse,
               concat,filter,sconcat,at,lstostr,Y,
               number,boolean,string,list}
}

\lstset{style=minilisp,language=Minilisp,mathescape}
\lstMakeShortInline[columns=fixed]@

\usepackage{bussproofs}

\newcommand{\Axi}[1]{\AxiomC{$#1$}}
\newcommand{\BInf}[1]{\BinaryInfC{$#1$}}
\newcommand{\UInf}[1]{\UnaryInfC{$#1$}}
\newcommand{\TInf}[1]{\TrinaryInfC{$#1$}}
\newcommand{\QInf}[1]{\QuaternaryInfC{$#1$}}
\newcommand{\QqInf}[1]{\QuinaryInfC{$#1$}}

\begin{document}
  \section{Especificación Formal del Lenguaje}
  \textbf{Sintaxis Concreta}
  \begin{align*}
    \nt{expr} \pr & \nt{id}\\
              \gr & \nt{num}\\
              \gr & \nt{bool}\\
              % Nuevo
              % TODO: Elegir representación de listas
              % \gr & \nt{list}\\
              % Nuevo
              \gr & \nt{string}\\
              \gr & \cd{(}\nt{op}\ \nt{expr}\ \{\nt{expr}\}\cd{)}\\
              \gr & \cd{(let ([}\nt{id}\ \nt{expr}\cd{]}\ \{\cd{[}\nt{id}\ \nt{expr}\cd{]}\}\ \nt{expr}\cd{)}\\
              \gr & \cd{(let* ([}\nt{id}\ \nt{expr}\cd{]}\ \{\cd{[}\nt{id}\ \nt{expr}\cd{]}\}\ \nt{expr}\cd{)}\\
              \gr & \cd{(if}\ \nt{expr}\ \nt{expr}\ \nt{expr}\cd{)}\\
              \gr & \cd{(cond (}\{\cd{[}\nt{expr}\ \nt{expr}\cd{]}\}\cd{) (else}\ \nt{expr}\cd{))}\\
              \gr & \cd{(lambda (}\nt{id}\ \{\nt{id}\}\cd{)}\ \nt{expr}\cd{)}\\
              \gr & \cd{(}\nt{expr}\ \nt{expr}\ \{\nt{expr}\}\cd{)}\\
              % Nuevo
              \gr & \cd{(letrec (}\nt{id}\ \nt{expr}\cd{)}\ \nt{expr}\cd{)}\\
              % Nuevo
              \gr & \cd{(list}\ \{\nt{expr}\}\cd{)}\\
              \text{\phantom{.}}\\
              %
    \nt{num} \pr & \dots \ \cd{2.5} \gr \cd{-1} \gr \cd{0} \gr \cd{1} \gr \cd{18.35} \ \dots\\
              \text{\phantom{.}}\\
              %
    \nt{bool} \pr & \cd{#t} \gr \cd{#f}\\
              \text{\phantom{.}}\\
              %
    % Nuevo
    % TODO: Elegir representación de listas
    % \nt{list} \pr & \cd{(}\{\nt{expr}\}\cd{)}\\
    %           \text{\phantom{.}}\\
              %
    \nt{string} \pr & \cd{"a"} \gr \cd{"b"} \gr \cd{"Hello world!"} \gr \dots\\
              \text{\phantom{.}}\\
              %
    \nt{op} \pr & \cd{+} \gr \cd{-} \gr \cd{*} \gr \cd{/} \gr \cd{add1} \gr \cd{sub1} \gr \cd{sqrt} \gr \cd{expt} \gr\\
                & \cd{<} \gr \cd{>} \gr \cd{=} \gr \cd{not} \gr \cd{or} \gr \cd{and} \gr\\
                & \cd{head} \gr \cd{tail} \gr \cd{length} \gr \cd{reverse} \gr \cd{concat} \gr \cd{map} \gr \cd{filter} \gr\\
                & \cd{sconcat} \gr \cd{at} \gr \cd{lstostr}\\
              \text{\phantom{.}}\\
              %
    \nt{id} \pr & \cd{a} \gr \cd{b} \gr \cd{foo} \gr \dots
  \end{align*}
  \clearpage

  \textbf{Sintaxis Abstracta Endulzada}
  \begin{center}
    $Ops = \{\cd{+},\,\cd{-},\,\cd{*},\,\cd{/},\,\cd{add1},\,\cd{sub1},\,\cd{sqrt},\,\cd{expt},\,
              \cd{<},\,\cd{>},\,\cd{=},\, \cd{not},\,\cd{or},\,\cd{and},$
              \hphantom{1.5cm} $\cd{head},\,\cd{tail},\,\cd{length},\,\cd{reverse},\,\cd{concat},\,\cd{map},\,\cd{filter}$\\
              \hphantom{1.5cm} $\cd{sconcat},\,\cd{at},\,\cd{lstostr}\}$
  \end{center}
  \centerline{$\texttt{Binding} \subseteq \texttt{String} \times \texttt{SASA}$}
  \begin{multicols}{3}
    \begin{prooftree}
      \Axi{i \texttt{:String}}
      \UInf{IdS(i) \texttt{:SASA}}
    \end{prooftree}

    \begin{prooftree}
      \Axi{n \in \mathbb{Z}}
      \UInf{NumS(n) \texttt{:SASA}}
    \end{prooftree}

    \begin{prooftree}
      \Axi{b \in \mathbb{B}}
      \UInf{BooleanS(b) \texttt{:SASA}}
    \end{prooftree}

    \begin{prooftree}
      \Axi{f \in Ops}
      \Axi{args\texttt{:[SASA]}}
      \BInf{Op(f,\,args)\texttt{:SASA}}
    \end{prooftree}

    \begin{prooftree}
      \Axi{b\texttt{:[Binding]}}
      \Axi{c\texttt{:SASA}}
      \BInf{Let(b,\,c)\texttt{:SASA}}
    \end{prooftree}

    \begin{prooftree}
      \Axi{b\texttt{:[Binding]}}
      \Axi{c\texttt{:SASA}}
      \BInf{Let1(b,\,c)\texttt{:SASA}}
    \end{prooftree}

    \begin{prooftree}
      \Axi{c\texttt{:SASA}}
      \Axi{t\texttt{:SASA}}
      \Axi{e\texttt{:SASA}}
      \TInf{If(c,\,t,\,e)\texttt{:SASA}}
    \end{prooftree}

    \begin{prooftree}
      \Axi{cs\texttt{:[(SASA, SASA)]}}
      \Axi{e\texttt{:SASA}}
      \BInf{Cond(cs,\,e)\texttt{:SASA}}
    \end{prooftree}

    \begin{prooftree}
      \Axi{p\texttt{:[String]}}
      \Axi{c\texttt{:SASA}}
      \BInf{Fun(p,\,c)\texttt{:SASA}}
    \end{prooftree}

    \begin{prooftree}
      \Axi{f\texttt{:SASA}}
      \Axi{a\texttt{:[SASA]}}
      \BInf{App(f,\,a)\texttt{:SASA}}
    \end{prooftree}

    \begin{prooftree}
      \Axi{i\texttt{:String}}
      \Axi{v\texttt{:SASA}}
      \Axi{c\texttt{:SASA}}
      \TInf{Letrec(i,\,v,\,c)\texttt{:SASA}}
    \end{prooftree}

    \begin{prooftree}
      \Axi{l\texttt{:[SASA]}}
      \UInf{List(l)\texttt{:SASA}}
    \end{prooftree}

    \begin{prooftree}
      \Axi{s \texttt{:String}}
      \UInf{StringS(s) \texttt{:SASA}}
    \end{prooftree}
  \end{multicols}

  \textbf{Sintaxis Abstracta Desendulzada}
  \begin{center}
    $U = \{\cd{add1},\,\cd{sub1},\,\cd{sqrt},\,\cd{not},\,\cd{head},\,\cd{lstostr}\}$

    $B = \{\cd{+},\,\cd{-},\,\cd{*},\,\cd{/},\,\cd{expt},\,\cd{<},\,\cd{>},\,\cd{=},\,\cd{or},\,\cd{and},\,\cd{map},\,\cd{filter},\,\cd{sconcat},\,\cd{at}\}$

    $U_{ns} = \{\cd{tail},\,\cd{length},\,\cd{reverse}\}$

    $B_{ns} = \{\cd{concat}\}$
  \end{center}
  \begin{multicols}{3}
    \begin{prooftree}
      \Axi{i\texttt{:String}}
      \UInf{Id(i)\texttt{:ASA}}
    \end{prooftree}

    \begin{prooftree}
      \Axi{n \in \mathbb{Z}}
      \UInf{Num(n)\texttt{:ASA}}
    \end{prooftree}

    \begin{prooftree}
      \Axi{b \in \mathbb{B}}
      \UInf{Boolean(b)\texttt{:ASA}}
    \end{prooftree}

    \begin{prooftree}
      \Axi{f \in U}
      \Axi{arg\texttt{:ASA}}
      \BInf{Unop(f,\, arg)\texttt{:ASA}}
    \end{prooftree}

    \begin{prooftree}
      \Axi{f \in B}
      \Axi{i\texttt{:ASA}}
      \Axi{d\texttt{:ASA}}
      \TInf{Binop(f,\,i,\,d)\texttt{:ASA}}
    \end{prooftree}

    \begin{prooftree}
      \Axi{c\texttt{:ASA}}
      \Axi{t\texttt{:ASA}}
      \Axi{e\texttt{:ASA}}
      \TInf{If(c,\,t,\,e)\texttt{:ASA}}
    \end{prooftree}

    \begin{prooftree}
      \Axi{p\texttt{:String}}
      \Axi{c\texttt{:ASA}}
      \BInf{Fun(p,\,c)\texttt{:ASA}}
    \end{prooftree}

    \begin{prooftree}
      \Axi{f\texttt{:ASA}}
      \Axi{a\texttt{:ASA}}
      \BInf{App(f,\,a)\texttt{:ASA}}
    \end{prooftree}

    \begin{prooftree}
      \Axi{l\texttt{:[ASA]}}
      \UInf{List(l)\texttt{:ASA}}
    \end{prooftree}

    \begin{prooftree}
      \Axi{s \texttt{:String}}
      \UInf{String(s)\texttt{:ASA}}
    \end{prooftree}
  \end{multicols}

  Valores finales
  \begin{multicols}{3}
    \begin{prooftree}
      \Axi{n \in \mathbb{Z}}
      \UInf{NumV(n)\texttt{:Value}}
    \end{prooftree}

    \begin{prooftree}
      \Axi{b \in \mathbb{B}}
      \UInf{BooleanV(b)\texttt{:Value}}
    \end{prooftree}

    \begin{prooftree}
      \Axi{p\texttt{:String}}
      \Axi{c\texttt{:Value}}
      \Axi{\amb\texttt{:Env}}
      \TInf{\nt{p,\,c,\,\amb}\texttt{:Value}}
    \end{prooftree}

    \begin{prooftree}
      \Axi{e\texttt{:ASA}}
      \Axi{\amb\texttt{:Env}}
      \BInf{\nt{e,\,\amb}\texttt{:Value}}
    \end{prooftree}

    \begin{prooftree}
      \Axi{l\texttt{:[Value]}}
      \UInf{ListV(l)\texttt{:Value}}
    \end{prooftree}

    \begin{prooftree}
      \Axi{s \texttt{:String}}
      \UInf{StringV(s)\texttt{:Value}}
    \end{prooftree}
  \end{multicols}

  \textbf{Semántica Natural}

  Identificadores se buscan en el ambiente y se retorna error de variable libre si no son encontrados
  \begin{prooftree}
    \Axi{}
    \UInf{Id(i),\amb \raa \amb(i)}
  \end{prooftree}

  Numeros se reducen a si mismos
  \begin{prooftree}
    \Axi{}
    \UInf{Num(n),\amb \raa NumV(n)}
  \end{prooftree}

  Booleanos se reducen a si mismos
  \begin{prooftree}
    \Axi{}
    \UInf{Boolean(b),\amb \raa BooleanV(b)}
  \end{prooftree}

  Listas se reducen a si mismas
  \begin{prooftree}
    \Axi{}
    \UInf{List(l),\amb \raa ListV(l)}
  \end{prooftree}

  Strings se reducen a si mismas
  \begin{prooftree}
    \Axi{}
    \UInf{String(s),\amb \raa StringV(s)}
  \end{prooftree}

  Las operaciones unarias y binarias que requieren puntos estrictos ($U,\,B$) se interpretan mediante la operación correspondiente
  \begin{prooftree}
    \Axi{elige(f) = g}
    \Axi{arg,\amb \raa a}
    \Axi{strict(a) = v'}
    \Axi{g(v') = v''}
    \QInf{Unop(f,\,arg),\amb \raa v''}
  \end{prooftree}

  \begin{prooftree}
    \Axi{elige(f) = g}
    \Axi{i,\amb \raa i' \quad d,\amb \raa d'}
    \Axi{strict(i') = i'' \quad strict(d') = d''}
    \Axi{g(i'', d'') = v}
    \QInf{Binop(f,i,d),\amb \raa v}
  \end{prooftree}
  Donde $elige$ es una función que transforma la operación en sintaxis abstracta a la operación correspondiente en
  el lenguaje anfitrión

  Las operaciones unarias y binarias que \textbf{no} requieren puntos estrictos ($U_{ns},\,B_{ns}$) se interpretan mediante la operación correspondiente
  \begin{prooftree}
    \Axi{elige(f) = g}
    \Axi{arg,\amb \raa a}
    \Axi{g(a) = v''}
    \TInf{Unop_{ns}(f,\,arg),\amb \raa v''}
  \end{prooftree}

  \begin{prooftree}
    \Axi{elige(f) = g}
    \Axi{i,\amb \raa i' \quad d,\amb \raa d'}
    \Axi{g(i', d') = v}
    \TInf{Binop_{ns}(f,i,d),\amb \raa v}
  \end{prooftree}
  Donde $elige$ es una función que transforma la operación en sintaxis abstracta a la operación correspondiente en
  el lenguaje anfitrión

  Condicional @if@
  \begin{prooftree}
    \Axi{c,\amb \raa c'}
    \Axi{strict(c') = Boolean(True)}
    \Axi{t,\amb \raa t'}
    \TInf{If(c,t,e),\amb \raa t'}
  \end{prooftree}

  \begin{prooftree}
    \Axi{c,\amb \raa c'}
    \Axi{strict(c') = Boolean(False)}
    \Axi{e,\amb \raa e'}
    \TInf{If(c,t,e),\amb \raa e'}
  \end{prooftree}

  Expresiones @lambda@
  \begin{prooftree}
    \Axi{}
    \UInf{Fun(p,c),\amb \raa \nt{p,c,\amb}}
  \end{prooftree}

  Aplicaciones de función
  \begin{prooftree}
    \Axi{f,\amb \raa f'}
    \Axi{strict(f') = \nt{p,c,\amb'}}
    \Axi{c,\amb'[p \gets \nt{a,\amb}] \raa c_v}
    \TInf{App(f, a),\amb \raa c_v}
  \end{prooftree}

  Notemos que el @letrec@ se convertirá en una aplicación de función con el combinador Y

  \begin{center}
  \scalebox{0.85}{
  \begin{minipage}{\linewidth}
  \begin{center}
    $\cd{(letrec (ft (lambda (n) (if (= n 0) 1 (* n (ft (- n 1)))))) (ft 5))}\ \raa$\\
    $\cd{((lambda (ft) (ft 5)) (Y (lambda (ft) (lambda (n) (if (= n 0) 1 (* n (ft (- n 1))))))))}$\\
    Lo cuál es igual a
    $\cd{((Y (lambda (ft) (lambda (n) (if (= n 0) 1 (* n (ft (- n 1))))))) 5)}$\\
    $ft = \lambda ft.\lambda n.\,\text{if}\ n = 0\ \text{then}\ 1\ \text{else}\ n \cdot ft(n-1)$\\
    $(Y\ ft)\ 5$
  \end{center}
  \end{minipage}
  }
  \end{center}


  \section{Algoritmo de Inferencia de Tipos}

  El \textbf{Algoritmo de Inferencia de Tipos} es un proceso que permite deducir el tipo de las expresiones en un programa sin necesidad de anotaciones explícitas. Uno de los algoritmos más utilizados es el \textbf{Algoritmo W}, asociado al sistema de tipos Hindley-Milner \cite{damas1982principal}.

  \subsection{Descripción del Algoritmo W}

  El Algoritmo W funciona recorriendo el árbol sintáctico abstracto del programa y generando un conjunto de ecuaciones de tipos (también conocidas como restricciones de tipos). Luego, resuelve estas ecuaciones mediante unificación, encontrando el tipo más general que satisface todas las restricciones \cite{pierce2002types}.

  \subsection{Restricciones}
  Identificadores\\
  Igualar todas las apariciones\\
  $\typx{x_1} = \typx{x_2} = \dots = \typx{x_n}$\\
  Que en realidad genera\\
  $\typx{x_1} = \typx{x_2}$\\
  $\dots$\\
  $\typx{x_1} = \typx{x_n}$

  Números\\
  $\typx{n} = \cd{number}$

  Booleanos\\
  $\typx{b} = \cd{boolean}$

  Strings\\
  $\typx{s} = \cd{string}$

  Listas\\
  $\typx{l} = \cd{list}$

  Operaciones aritméticas\\
  $op = \{\cd{add1},\,\cd{sub1},\,\cd{sqrt},\,\cd{+},\,\cd{-},\,\cd{*},\,\cd{/},\,\cd{expt}\}$\\
  $\typx{\cd{(}op\ n_1\ \dots\ n_i\cd{)}} = \cd{number}$\\
  $\typx{n_1} = \cd{number}$\\
  $\dots$\\
  $\typx{n_i} = \cd{number}$

  Comparaciones\\
  $op = \{\cd{<},\,\cd{>},\,\cd{=}\}$\\
  $\typx{\cd{(}op\ n_1\ \dots\ n_i\cd{)}} = \cd{boolean}$\\
  $\typx{n_1} = \cd{number}$\\
  $\dots$\\
  $\typx{n_i} = \cd{number}$

  Operaciones booleanas\\
  $op = \{\cd{not},\,\cd{or},\,\cd{and}\}$\\
  $\typx{\cd{(}op\ b_1\ \dots\ b_n\cd{)}} = \cd{boolean}$\\
  $\typx{b_1} = \cd{boolean}$\\
  $\dots$\\
  $\typx{b_n} = \cd{boolean}$

  If\\
  $\typx{\cd{(if}\ c\ t\ e\cd{)}} = \typx{t}$\\
  $\typx{\cd{(if}\ c\ t\ e\cd{)}} = \typx{e}$\\
  $\typx{c} = \cd{boolean}$\\
  $\typx{t} = \typx{e}$
  
  \clearpage

  Cond\\
  $\typx{\cd{(cond ([}c_1\ t_1\cd{]}\ \dots\ \cd{[}c_n\ t_n\cd{]) (else}\ e\cd{))}} = \typx{t_1}$\\
  $\dots$\\
  $\typx{\cd{(cond ([}c_1\ t_1\cd{]}\ \dots\ \cd{[}c_n\ t_n\cd{]) (else}\ e\cd{))}} = \typx{t_n}$\\
  $\typx{\cd{(cond ([}c_1\ t_1\cd{]}\ \dots\ \cd{[}c_n\ t_n\cd{]) (else}\ e\cd{))}} = \typx{e}$\\
  $\typx{c_1} = \cd{boolean}$\\
  $\dots$\\
  $\typx{c_n} = \cd{boolean}$\\
  $\typx{t_1} = \typx{t_2}$\\
  $\dots$\\
  $\typx{t_1} = \typx{t_n}$\\
  $\typx{t_1} = \typx{e}$

  Let\\
  $\typx{\cd{(let ([}i_1\ v_1\cd{]}\ \dots\ \cd{[}i_n\ v_n\cd{]}\cd{)}\ c\cd{)}} = \typx{c}$\\
  $\typx{i_1} = \typx{v_1}$\\
  $\dots$\\
  $\typx{i_n} = \typx{v_n}$

  % TODO: Hacerlo bien
  Let*\\
  $\typx{\cd{(let* ([}i_1\ v_1\cd{]}\ \dots\ \cd{[}i_n\ v_n\cd{]}\cd{)}\ c\cd{)}} = \typx{c}$\\
  $\typx{i_1} = \typx{v_1}$\\
  $\dots$\\
  $\typx{i_n} = \typx{v_n}$

  % TODO: Letrec

  Funciones\\
  $\typx{\cd{(lambda (}p_1\ \dots\ p_n\cd{)}\ b\cd{)}} = \typx{p_1} \rightarrow \dots \rightarrow \typx{p_n} \rightarrow \typx{b}$

  Aplicaciones de función $\cd{(}f\ \ a_1\ \dots\ a_n\cd{)}$\\
  $\typx{f} = \typx{a_1} \rightarrow \dots \rightarrow \typx{a_n} \rightarrow \typx{\cd{(}f\ \ a_1\ \dots\ a_n\cd{)}}$

  Concatenación de Strings\\
  $\typx{\cd{(sconcat}\ s_1\ \dots\ s_n\cd{)}} = \cd{string}$\\
  $\typx{s_1} = \cd{string}$\\
  $\dots$\\
  $\typx{s_n} = \cd{string}$

  At (Obtener cadena en un índice)\\
  $\typx{\cd{(at}\ n\ s\cd{)}} = \cd{string}$\\
  $\typx{n} = \cd{number}$\\
  $\typx{s} = \cd{string}$

  Head\\
  $\typx{\cd{(head}\ l\cd{)}} = T_{uuid}$\\
  $\typx{l} = \cd{list}$

  Tail\\
  $\typx{\cd{(tail}\ l\cd{)}} = \cd{list}$\\
  $\typx{l} = \cd{list}$

  Length\\
  $\typx{\cd{(length}\ l\cd{)}} = \cd{number}$\\
  $\typx{l} = \cd{list}$

  Reverse\\
  $\typx{\cd{(reverse}\ l\cd{)}} = \cd{list}$\\
  $\typx{l} = \cd{list}$

  Concatenación\\
  $\typx{\cd{(concat}\ l_1\ l_2\cd{)}} = \cd{list}$\\
  $\typx{l_1} = \cd{list}$\\
  $\typx{l_2} = \cd{list}$

  Lista a String\\
  $\typx{\cd{(lstostr}\ l\cd{)}} = \cd{string}$\\
  $\typx{l} = \cd{list}$

  Filter\\
  $\typx{\cd{(filter}\ f\ l\cd{)}} = \cd{list}$\\
  $\typx{f = \cd{(lambda (}p\cd{)}\ b\cd{)}} = \typx{p} \rightarrow \typx{b}$\\
  $\typx{b} = \cd{boolean}$\\
  $\typx{l} = \cd{list}$

  Map\\
  $\typx{\cd{(map}\ f\ l\cd{)}} = \cd{list}$\\
  $\typx{f = \cd{(lambda (}p\cd{)}\ b\cd{)}} = \typx{p} \rightarrow \typx{b}$\\
  $\typx{l} = \cd{list}$

\end{document}
